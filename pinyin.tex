% !Mode:: "TeX::UTF-8"
\documentclass[oneside]{book}

% ========== 基础宏包引入(必备功能) ========== %
\usepackage[a4paper,
  bindingoffset=10mm,
  top=35mm,  
  bottom=30mm,
  inner=25mm,  % 优化:增大左页边距,避免贴装订线
  outer=15mm,  % 优化:右页边距适度调整
  headheight=10mm,
  headsep=15mm,
  footskip=15mm,
  marginparsep=0pt,
  marginparwidth=0em,
  includemp=true
  ]{geometry}

\usepackage[UTF8, heading=true, scheme=chinese, fontset=none]{ctex}
\ctexset{fontset = fandol}

% 补充缺失宏包,修复兼容性
\usepackage{fontspec} % 支持\fontspec命令
\usepackage{microtype}
\usepackage{setspace}
\usepackage{hyperref}
\usepackage{xcolor}
\usepackage{xpinyin}
\usepackage{tikz}
\usepackage{array}
\usepackage{fancyhdr} % 新增:页码配置

% ========== 样式配置(统一管理,修改高效) ========== %
% 1. 表格框:边框+圆角
\newcommand{\tableBox}[1]{
  \begin{tikzpicture}
    \node[draw, inner sep=8pt, rounded corners=5pt, line width=1pt] {#1};
  \end{tikzpicture}
}

% 2. 纯拼音表格样式
\newcommand{\pinyinTable}[1]{
  \renewcommand{\arraystretch}{3}
  \xpinyinsetup{
    font={\fontspec{DejaVu Sans Mono}\fontsize{36pt}{42pt}\selectfont},
  }
  \begin{center}
  \tableBox{#1}
  \end{center}
}

% 3. 汉字+拼音表格样式
\newcommand{\charPinyinTable}[1]{
  \renewcommand{\arraystretch}{5}
  \xpinyinsetup{
    font={\fontspec{DejaVu Sans Mono}\selectfont},
    ratio={1},
    hsep={.6em},
    vsep={1em}
  }
  \begin{center}
  \tableBox{#1}
  \end{center}
}

% 4. 大汉字字体
\newcommand{\charFont}{\fontsize{32pt}{38pt}\selectfont}

% 快捷拼音+汉字组合(更易维护)
\newcommand{\pybad}[1]{\strike{\pinyin{#1}}}

% ========== 自定义命令(功能扩展) ========== %
% 优化删除线:2pt粗细,不遮挡文字
\newcommand{\strike}[1]{%
  \tikz[baseline=(text.base)]{
    \node[inner sep=0pt, outer sep=0pt] (text) {#1};
    \draw[red, line width=2pt] (text.south west) -- (text.north east);
  }%
}

% 超链接配置
\hypersetup{
  colorlinks=true,
  linkcolor=blue,
  citecolor=green,
  filecolor=magenta,
  urlcolor=cyan
}

% 页码配置:页脚居中
\pagestyle{fancy}
\fancyhf{}
\fancyfoot[C]{\thepage}

% ========== 文档信息 ========== %
\title{拼音表}
\author{田利建}
\date{\today}

\begin{document}
\maketitle

% ========== 正文内容 ========== %
\section{b、p、m、f}
  \pinyinTable{
    \begin{tabular}{c@{\hspace{3em}}c@{\hspace{3em}}c@{\hspace{3em}}c@{\hspace{3em}}c@{\hspace{3em}}c}
      \pinyin{ba} & \pinyin{bo} & \pybad{be} & \pinyin{bi} & \pinyin{bu} & \strike{\pinyin{bü}} \\
      \pinyin{pa} & \pinyin{po} & \pybad{pe} & \pinyin{pi} & \pinyin{pu} & \strike{\pinyin{pü}} \\
      \pinyin{ma} & \pinyin{mo} & \pinyin{me} & \pinyin{mi} & \pinyin{mu} & \strike{\pinyin{mü}} \\
      \pinyin{fa} & \pinyin{fo} & \pybad{fe} & \pybad{fi} & \pinyin{fu} & \strike{\pinyin{fü}} \\
    \end{tabular}
  }

  \vspace{6em}

  \charPinyinTable{
    % \begin{tabular}{>{\charFont}c@{\hspace{3em}}>{\charFont}c@{\hspace{3em}}>{\charFont}c@{\hspace{3em}}>{\charFont}c@{\hspace{3em}}>{\charFont}c@{\hspace{3em}}>{\charFont}c}
    \begin{tabular}{*{5}{>{\charFont}c@{\hspace{3em}}} >{\charFont}c} % 前5列:带3em列间隔
      \xpinyin{八}{ba} & \xpinyin{波}{bo} & \pybad{be} & \xpinyin{比}{bi} & \xpinyin{不}{bu} & \strike{\pinyin{bü}} \\
      \xpinyin{怕}{pa} & \xpinyin{破}{po} & \pybad{pe} & \xpinyin{皮}{pi} & \xpinyin{普}{pu} & \strike{\pinyin{pü}} \\
      \xpinyin{妈}{ma} & \xpinyin{陌}{mo} & \xpinyin{么}{me} & \xpinyin{米}{mi} & \xpinyin{木}{mu} & \strike{\pinyin{mü}} \\
      \xpinyin{发}{fa} & \xpinyin{佛}{fo} & \pybad{fe} & \pybad{fi} & \xpinyin{夫}{fu} & \strike{\pinyin{fü}} \\
    \end{tabular}
  }


\section{d、t、n、l}

  \pinyinTable{
    \begin{tabular}{c@{\hspace{3em}}c@{\hspace{3em}}c@{\hspace{3em}}c@{\hspace{3em}}c@{\hspace{3em}}c}
      \pinyin{da} & \pybad{do} & \pinyin{de} & \pinyin{di} & \pinyin{du} & \strike{\pinyin{dü}} \\
      \pinyin{ta} & \pybad{to} & \pinyin{te} & \pinyin{ti} & \pinyin{tu} & \strike{\pinyin{tü}} \\
      \pinyin{na} & \pybad{no} & \pinyin{ne} & \pinyin{ni} & \pinyin{nu} & \pinyin{nü} \\
      \pinyin{la} & \pybad{lo} & \pinyin{le} & \pinyin{li} & \pinyin{lu} & \pinyin{lü} \\
    \end{tabular}
  }

  \vspace{6em}

  \charPinyinTable{
    \begin{tabular}{*{5}{>{\charFont}c@{\hspace{3em}}} >{\charFont}c} % 前5列:带3em列间隔
      \xpinyin{大}{da} & \pybad{do} & \xpinyin{的}{de} & \xpinyin{地}{di} & \xpinyin{读}{du} & \strike{\pinyin{dü}} \\
      \xpinyin{他}{ta} & \pybad{to} & \xpinyin{特}{te} & \xpinyin{体}{ti} & \xpinyin{土}{tu} & \strike{\pinyin{tü}} \\
      \xpinyin{那}{na} & \pybad{no} & \xpinyin{呢}{ne} & \xpinyin{你}{ni} & \xpinyin{努}{nu} & \xpinyin{女}{nü} \\
      \xpinyin{拉}{la} & \pybad{lo} & \xpinyin{乐}{le} & \xpinyin{利}{li} & \xpinyin{路}{lu} & \xpinyin{旅}{lü} \\
    \end{tabular}
  }

\section{g、k、h}
  \pinyinTable{
    \begin{tabular}{*{5}{c@{\hspace{3em}}} c} % 前5列:默认对齐(c) + 3em间隔
      \pinyin{ga} & \pybad{go} & \pinyin{ge} & \pybad{gi} & \pinyin{gu} & \strike{\pinyin{gü}} \\
      \pinyin{ka} & \pybad{ko} & \pinyin{ke} & \pybad{ki} & \pinyin{ku} & \strike{\pinyin{kü}} \\
      \pinyin{ha} & \pybad{ho} & \pinyin{he} & \pybad{hi} & \pinyin{hu} & \strike{\pinyin{hü}} \\
    \end{tabular}
  }

  \vspace{6em}

  \charPinyinTable{
    \begin{tabular}{*{5}{>{\charFont}c@{\hspace{3em}}} >{\charFont}c} % 前5列:带3em列间隔
      \xpinyin{嘎}{ga} & \pybad{go} & \xpinyin{个}{ge} & \pybad{gi} & \xpinyin{古}{gu} & \strike{\pinyin{gü}} \\
      \xpinyin{卡}{ka} & \pybad{ko} & \xpinyin{可}{ke} & \pybad{ki} & \xpinyin{苦}{ku} & \strike{\pinyin{kü}} \\
      \xpinyin{哈}{ha} & \pybad{ho} & \xpinyin{和}{he} & \pybad{hi} & \xpinyin{胡}{hu} & \strike{\pinyin{hü}} \\
    \end{tabular}
  }
\end{document}
